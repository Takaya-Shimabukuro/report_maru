\documentclass[a4paper,11pt]{bxjsarticle}
\usepackage{xltxtra}
\usepackage{zxjatype}
\usepackage{here}
\setjamainfont[BoldFont=ipaexm.ttf]{ipaexm.ttf}
\setjasansfont[BoldFont=ipaexg.ttf]{ipaexg.ttf}
\usepackage{enumitem}
\usepackage{amsmath}
\usepackage{amssymb}
\usepackage{booktabs}
\usepackage{listings}
\usepackage{bm}
\lstset{
basicstyle={\scriptsize}
}
\setlist[enumerate]{listparindent=2pt}


\begin{document}
\begin{titlepage}
  \begin{center}
    \vspace*{150truept}
    {\Huge マルチメディア工学Ⅰ 中間レポート}\\ % タイトル
    \vspace{120truept}
    {\huge 18T1694W}\\ % 学籍番号
    \vspace{50truept}
    {\huge 島袋 隆也}\\ % 著者
    \vspace{50truept}
    {\huge \today}\\ % 提出日
  \end{center}
\end{titlepage}


%=====================================================================
\section{目的}
 本実験では,撮影した画像の圧縮率を変化させ,圧縮による画像劣化を考察する.また,
PNGやGIFに変換し,保存形式による画質の変化を考察し理解することを目的とする.


%=====================================================================
\section{実験器具}
  今回撮影に使用したスマホのスペックを表に示す. 
  \begin{table}[htb]
    \begin{center}
      \caption{スペック表}
      \begin{tabular}{|c|c|} \hline
        名称 & Garaxy s10  \\
        F値 & 2.4  \\
        画素数 & 1200万  \\ \hline

      \end{tabular}
      \label{tab:price}
    \end{center}
  \end{table}

\section{原理}
\subsection{JPEG}
\subsection{PNG}
\subsection{GIF}

%=====================================================================
\section{課題}
課題内容を以下に示す.
 \begin{enumerate}
    \item デジカメ等で撮影したJPEG画像の圧縮率を変化させて保存した複数の画像を作成し,画質とデータサイズとの関係を考察しなさい。また,画質がどのように変化しているか。またなぜそのような変化をするのかを考察しなさい。
    \item 同じ画像をJPEG以外の画像フォーマットで保存し,JPEGとの比較考察をしなさい。
  \end{enumerate}



\section{実験手順}
実験器具で示したスマートフォンと画像編集ソフトIrfanViewで実験を行った.
実験手順を以下に示す.\\
\begin{enumerate}
  \item \begin{enumerate}
          \item スマホで写真を撮影する.
          \item IrfanViewで写真を開く.
          \item "ファイル" → "名前をつけて保存"を選択する.
          \item ファイルの種類を"JPG-JPEG File"にする.
          \item "save options"より"save quality"(圧縮率)を0~100の10刻みで実行する.
        \end{enumerate}
  \item \begin{enumerate}
          \item ファイルの種類を"PNG-Portable Network Graphics"にする.
          \item "save options"よりsave quality(圧縮率)を0\textasciitilde100の10刻みで実行する.
          \item ファイルの種類を"GIF-Compuserve GIF"にする.
          \item "save options"よりsave quality(圧縮率)を0\textasciitilde100の10刻みで実行する.
        \end{enumerate}
\end{enumerate}



%=====================================================================
\section{結果と考察}
圧縮率を0\textasciitilde100まで10\%刻みで行った.その結果を図\ref{}から図\ref{}に示す.
  \subsection{結果}
  \begin{enumerate}
    \item 
  \end{enumerate}

   

  \subsection{考察}
    


%=====================================================================
\section{参考文献}
京都大学工学部電気系教室: 「電気電子工学実習」 \\

\end{document}
